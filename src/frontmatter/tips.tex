\documentclass[../main.tex]{subfiles}
\begin{document}

% Ensure the appendix numbering matches the main document when compiling this file alone.
\renewcommand{\thesection}{A\arabic{section}}
\renewcommand{\thetable}{A\arabic{section}.\arabic{table}}
\counterwithin{table}{section}
\counterwithin{figure}{section}

% Uncomment the block below when using a class other than book.
%\renewcommand{\thesubsection}{A\arabic{subsection}}
%\renewcommand{\thetable}{A\arabic{subsection}.\arabic{table}}
%\counterwithin{table}{subsection}
%\counterwithin{figure}{subsection}

\definecolor{lightgray}{RGB}{247,247,247}

\section{Getting Started}
This appendix collects loose tips for maintaining the thesis project. It is not included in the compiled thesis unless you explicitly reference it from \textit{main.tex}. Compile the file on its own whenever you want to preview the notes.

\section{Adding Chapters and Sections}
The thesis is split into subfiles inside `src/chapters/`. Each chapter lives in its own \LaTeX{} file, which keeps the project tidy and lets you compile chapters independently.

To create a new chapter, add a file under `src/chapters/` and use the following template:

\linespread{1}
\begin{minted}[
  bgcolor=lightgray,
  style=vs
]{latex}
\documentclass[../main.tex]{subfiles}
\begin{document}

% Write your chapter content here

\biblio
\end{document}
\end{minted}
\linespread{1.3}
Remember to register the file in ``src/main.tex'' with ``\subfile{chapters/<filename>}'' so it appears in the final PDF in the desired order.

\subsection{Section Titles}
Create numbered subsections with ``\texttt{\subsection{\ldots}}''. Deeper levels are available through ``\texttt{\subsubsection}'' and ``\texttt{\paragraph}''. The counters follow the pattern Chapter.Section.Subsection.Paragraph.

\section{Figures, Tables, and Equations}
\subsection{Figures}
Place illustrative material under ``src/figures/''. Thanks to the configured ``\graphicspath'', it is enough to reference images by filename:

\linespread{1}
\begin{minted}[
  bgcolor=lightgray,
  style=vs
]{latex}
\begin{figure}[H]
    \centering
    \caption{University seal}
    \includegraphics[width=0.2\columnwidth]{escudo_udec.png}
    \label{fig:udec_logo}
\end{figure}
\end{minted}
\linespread{1.3}
The ``\texttt{\label}'' should follow the caption so cross references via ``\texttt{\ref{}}'' point to the correct number.

\subsection{Tables}
You can design tables manually or generate them with online tools such as \url{https://www.tablesgenerator.com/}. The snippet below produces Table~\ref{tab:variables-example}.

\linespread{1}
\begin{minted}[
  bgcolor=lightgray,
  style=vs
]{latex}
\begin{table}[H]
    \centering
    \caption{Variables in the dataset}
    \label{tab:variables-example}
    \begin{tabular}{lr}
        \hline\hline
        Variable name & Type \\
        \hline
        Postal code       & Categorical \\
        Date of birth     & Date \\
        Municipality      & Categorical \\
        Registration date & Date \\
        Total mass        & Numerical \\
        \hline\hline
    \end{tabular}
\end{table}
\end{minted}
\linespread{1.3}

\subsection{Equations}
Display equations using the equation environment so that numbering and referencing remain automatic:

\linespread{1}
\begin{minted}[
  bgcolor=lightgray,
  style=vs
]{latex}
\begin{equation}\label{eq:example}
    \hat{f}(x) = \sum_{b=1}^{B} \lambda \, \hat{f}^{(b)}(x)
\end{equation}
\end{minted}
\linespread{1.3}
Inline expressions can be written with ``\(\ldots\)'', e.g. ``\(x + x = y\)''.

\section{Appendix Numbering}
When compiling this file separately the section counters start with the prefix ``A''. Figures, tables, and equations inherit the same prefix to match the main document.

\section{References}
The project relies on the ``natbib'' package with the ``apalike'' style. Add bibliography entries to ``src/cites.bib'' and cite them in the text with ``\texttt{\citep}'' or ``\texttt{\citet}''. Use ``\texttt{\nocite{*}}'' if you need to print the entire bibliography for drafting purposes.

\section{Miscellaneous Tips}
Comments start with the percent symbol (``\%''). Escape literal percent signs with ``\texttt{\%}'' and dollar signs with ``\texttt{\$}''. For more \LaTeX{} examples, the community-driven documentation at \url{https://tex.stackexchange.com/} and the Overleaf guides are excellent references.

\biblio
\end{document}
