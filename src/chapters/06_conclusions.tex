\documentclass[../main.tex]{subfiles}
\begin{document}

\section{Conclusions and future work}
\label{chap:conclusions}

This chapter synthesises the principal outcomes of the thesis, outlines prospective avenues for further investigation, and reflects on the broader significance of the study.

\subsection{Summary of key findings}
\label{subsec:conclusions-summary}

The analysis presented herein delivers a comprehensive, \textit{Gaia}-based characterisation of NGC~6383. The main conclusions are:
\begin{enumerate}
  \item \textbf{Membership catalogue:} A probabilistic workflow combining HDBSCAN clustering and parallax consistency checks yields a fiducial sample of 254 high-confidence members, significantly improving completeness and purity compared with earlier catalogues.
  \item \textbf{Fundamental parameters:} Bayesian inference constrains the cluster distance to $1.11^{+0.06}_{-0.05}$~kpc, extinction to $A_V = 1.24^{+0.26}_{-0.22}$~mag, age to $3.5^{+1.4}_{-1.0}$~Myr, and metallicity to $\text{[Fe/H]} = +0.02 \pm 0.12$.
  \item \textbf{Structural properties:} The cluster exhibits a compact core ($R_c = 0.63 \pm 0.06$~pc), an extended tidal radius ($R_t = 13.0^{+4.5}_{-4.2}$~pc), and a concentration parameter of $C \approx 3.0$, placing it among moderately rich, centrally condensed young clusters.
  \item \textbf{Stellar populations:} The presence of 53 PMS candidates and a YSO fraction of $\sim28\%$ confirms ongoing or very recent star formation, while the CMD-based masses indicate a rich intermediate- and high-mass population.
  \item \textbf{Dynamical state:} Statistically significant mass segregation among stars with $M\gtrsim1.8~M_\odot$, alongside a half-mass relaxation time exceeding the cluster age, points to a primordial origin for the observed central concentration of massive (and likely binary) members.
\end{enumerate}

\subsection{Implications}
\label{subsec:conclusions-implications}

Collectively, these findings establish NGC~6383 as a benchmark system for probing the early dynamical evolution of open clusters in the \textit{Gaia} era. The results underscore the necessity of probabilistic membership methods when interpreting cluster CMDs, structural profiles, and mass functions. They also illustrate how combining machine-learning classifiers with Bayesian inference can disentangle complex astrophysical questions, such as the origin of mass segregation and the timeline of star formation.

\subsection{Future work}
\label{subsec:conclusions-future}

Several lines of enquiry emerge naturally from this study:
\begin{itemize}
  \item \textbf{Spectroscopic follow-up:} High-resolution radial-velocity monitoring of cluster members would refine the internal velocity dispersion, enable a dynamical mass estimate, and identify unresolved binary systems.
  \item \textbf{Deep photometry and infrared surveys:} Deeper near-infrared imaging (e.g.\ VISTA, JWST) could extend membership to sub-solar masses, probing the low-mass IMF and testing mass segregation among the faintest stars.
  \item \textbf{Time-domain observations:} Monitoring variability and accretion indicators in PMS stars would clarify the star-formation history and disk lifetimes within the cluster.
  \item \textbf{$N$-body simulations:} Tailored dynamical models could explore the survival prospects of NGC~6383 under various tidal environments, helping to determine whether it will remain bound or disperse over the next $\sim100$~Myr.
  \item \textbf{Comparative studies:} Applying the methodology to neighbouring clusters in the same star-forming complex (e.g.\ NGC~6530, Collinder~334) would reveal whether primordial mass segregation and high YSO fractions are common outcomes in similar environments.
\end{itemize}

\subsection{Final remarks}
\label{subsec:conclusions-final}

By uniting precise astrometry, machine-learning techniques, and Bayesian modelling, this thesis provides the most detailed portrait to date of NGC~6383. The approach exemplifies how \textit{Gaia} has transformed open-cluster studies from cataloguing exercises into laboratories for quantitative astrophysics. As future \textit{Gaia} releases, ground-based surveys, and theoretical models continue to mature, the methods and insights developed here will remain valuable for decoding the life cycles of stellar clusters across the Milky Way.

\biblio
\end{document}