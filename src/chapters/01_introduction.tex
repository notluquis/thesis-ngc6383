\documentclass[../main.tex]{subfiles}
\begin{document}

\section{Introduction}

Open clusters are fundamental laboratories for understanding stellar evolution and Galactic structure. As ensembles of stars born from the same molecular cloud, they share nearly uniform ages, initial chemical compositions, and distances. This common origin allows astronomers to isolate the effects of stellar mass, binarity, and environment on stellar evolution under controlled initial conditions. Their colour-magnitude diagrams (CMDs) capture entire populations at once, enabling precise determination of cluster distances, reddening, and ages through isochrone fitting. Moreover, because open clusters span a wide range of ages (from a few Myr to several Gyr) and reside throughout the Galactic disk, they serve as chronological markers of the Milky Way's star formation history and tracers of its structural and kinematic evolution. Young clusters, in particular, preserve the memory of their natal star-forming environment, while older clusters carry imprints of secular Galactic processes such as radial migration, disk heating, and tidal stripping.

Beyond their role in stellar evolution, open clusters provide insights into dynamical processes in gravitationally bound systems. The spatial and kinematic structure of a cluster encodes its dynamical state: measurements of stellar density profiles, core and tidal radii, and velocity dispersion distinguish compact, dynamically evolved clusters from looser associations still unrelaxed. Comparisons between a cluster's age and its two-body relaxation time indicate whether observed mass segregation (the concentration of massive stars toward the center) is likely primordial or the result of dynamical evolution. For instance, if high-mass stars are centrally concentrated in a cluster much younger than its relaxation timescale, this suggests primordial mass segregation imprinted at formation, whereas similar segregation in an older cluster could be a dynamical effect of energy equipartition. The binary fraction and binary orbital properties in clusters also offer clues: a high incidence of massive binaries can induce early mass segregation and hasten core collapse, influencing the cluster's survivability. Studying clusters at different ages thus illuminates how stellar systems dynamically relax, mass segregate, and eventually dissolve under the combined influence of internal N-body interactions and external tidal forces.

Accurate membership determination is the foundation of all such open cluster studies. Contamination by field stars can bias CMD analyses, inflate apparent age spreads, and mask kinematic trends. Historically, membership was inferred from photometry and proper motions using classical statistical methods \citep{2007A&A...462..157P}. Pioneering work by \citet{1930LicOB..14..154T} and \citet{1949ApJ...110..117S}, for example, relied on star counts and rudimentary proper-motion data to identify cluster members, while subsequent studies introduced more quantitative approaches such as photometric criteria \cite{2007A&A...462..157P}, radial-velocity selection, and astrometric probability calculations (e.g.\citealt{1978MNRAS.182..607F,1989MNRAS.236..263P,1991MNRAS.249...76B}). Each method had limitations, and in many cases they yielded discrepant member lists and cluster parameters. In the last decade, Bayesian inference techniques and machine-learning algorithms have emerged as powerful tools to refine membership probabilities and to combine multi-dimensional data (positions, proper motions, parallaxes, photometry) into a cohesive analysis \citep{2021scgr.confE..45H}. By probabilistically modeling the kinematics and photometry of cluster stars versus field stars, these approaches mitigate biases of earlier methods and allow robust membership catalogs that underpin reliable determination of cluster properties. In this thesis, we leverage such modern techniques to reassess the member census of the young open cluster NGC 6383 and thereby obtain more accurate measurements of its fundamental parameters.

\subsection{Open clusters as laboratories for Galactic astrophysics}\label{sec:openclusters}

Because of their coeval nature and common origin, open clusters are ideal testbeds for stellar evolution models. Within a single cluster, stars span a range of masses but share the same age and initial chemical composition; the morphology of their CMD (from the brightest massive stars down to low-mass dwarfs and pre-main-sequence objects) can thus be directly compared to theoretical isochrones to validate stellar physics (e.g. convective overshooting, rotation, and mixing processes). Any deviation of cluster stars from a single isochrone may indicate phenomena like differential rotation rates or unresolved binaries, or in very young clusters, could hint at an age spread or prolonged star formation. Furthermore, clusters provide critical calibrators for the luminosity-mass-age relation: for example, the turn-off mass of an open cluster defines the stellar mass that is just exhausting core hydrogen at the cluster's age, anchoring the high-mass end of stellar lifetimes.

In a Galactic context, the ensemble of known open clusters traces the Milky Way's thin disk structure and chemical evolution \citep{2020A&A...640A...1C}. Clusters form in the disk's molecular clouds and are initially embedded in their natal gas; the survival and dispersal of clusters are linked to Galactic environment (spiral arm passages, tidal forces, etc.). Young clusters (ages $\lesssim 10$ Myr) still reside near their birthplaces and often within star-forming regions, offering snapshots of recent star formation conditions including initial mass functions and star formation efficiency. Intermediate-age clusters (hundreds of Myr) have drifted from their birth sites and can be used to map the Galactic metallicity gradient and the vertical distribution of the disk, since they are old enough to have experienced orbital oscillations. The oldest open clusters (several Gyr) are rare (most disperse by then) but record early disk conditions and serve as tracers of the Milky Way's early chemical enrichment. By analyzing cluster populations as a function of age and location, astronomers can probe phenomena such as radial migration (clusters moving from their birth radii), the disk's secular heating (increasing velocity dispersion with age), and the survival rate of clusters over time \citep{2020A&A...640A...1C,2020A&A...633A..99C}.

Young open clusters containing massive stars also shed light on feedback processes. Massive O- and B-type members produce strong UV radiation fields and stellar winds that can disrupt or trigger star formation in the surrounding molecular material. Clusters like NGC 3603 and Westerlund 2, for instance, illustrate how OB star feedback can both photoevaporate residual gas and compress nearby cloud regions to form new stars. In more modest clusters, even a single O-star can have significant influence: in NGC 6383, the O7V + O7V binary HD 159176 ionizes the surrounding \ion{H}{ii} region and was hypothesized to have triggered sequential star formation (see Section \ref{sec:NGC6383}). Thus, studying the spatial distribution of young stellar objects (YSOs) and gas around clusters allows us to investigate triggered star formation and feedback-regulated cluster growth \citep{2007A&A...462..157P}.

Finally, open clusters provide a unique opportunity to study stellar dynamics on small scales. Within a cluster, stars interact gravitationally over time, leading to energy exchange and mass segregation. By comparing observations of clusters at different ages, one can test N-body simulation predictions of cluster dynamical evolution. For example, observations often show that the most massive stars in a young cluster are more centrally concentrated than lower-mass stars (mass segregation). In very young clusters (younger than their relaxation time), this is thought to reflect initial conditions ("primordial" mass segregation), whereas in older clusters it likely results from dynamical evolution (dynamical mass segregation). Measuring mass segregation indices and mass-dependent radial distributions in clusters like NGC 6383 offers insight into whether the cluster was born mass-segregated or has begun to relax and segregate via two-body interactions. Additionally, by identifying binaries and high-order multiples in clusters, we can assess how dynamical interactions (like binary hardening or exchanges) shape the cluster's internal energy balance. Open clusters therefore bridge stellar and Galactic scales: they help calibrate stellar astrophysics and at the same time exemplify the interplay between star formation, stellar dynamics, and Galactic environment.

\subsection{The \textit{Gaia} revolution in open cluster studies}\label{sec:gaia}

The European Space Agency's \textit{Gaia} mission has transformed the study of open clusters by providing precision astrometry (parallaxes and proper motions) and photometry for nearly all cluster members down to faint magnitudes across the entire sky. Prior to \textit{Gaia}, cluster membership was often established using only photometric criteria or relatively imprecise proper motions from ground-based catalogs, which limited the reliability of member lists, especially for distant or sparse clusters. The advent of \textit{Gaia} Data Release 2 (DR2) and Data Release 3 (DR3) has enabled membership determination in multi-dimensional astrometric space, sharply reducing field-star contamination. Stars in a true cluster will exhibit a tight clustering in proper motion space and a narrow parallax distribution, distinct from the field population. By selecting stars that share common proper motion vectors and parallaxes, one can identify cluster members over large areas of sky, including far from the cluster core where traditional photometric selection would struggle \citep{2021scgr.confE..45H,2018A&A...620A..89N}. Indeed, analyses of \textit{Gaia} DR2 led to the discovery of hundreds of new open clusters and the revision of membership for many known clusters \citep{2020A&A...640A...1C,2022A&A...660A...4H}. For example, \citet{2020A&A...640A...1C} compiled an all-sky cluster catalog from \textit{Gaia} DR2, unveiling new sparse groups and halo structures, and showed that many classical clusters have larger spatial extents than previously recognized once faint members are included. Similarly, \citet{2021MNRAS.504..356D} used DR2 data to homogenize the parameters of 1743 clusters, illustrating the power of Gaia to provide a global, uniform cluster characterization.

The exquisite \textit{Gaia} astrometry (parallax uncertainties down to $\sim10\sim\mu$as for bright stars in DR3, and proper motions accurate to $\sim0.01$ mas yr$^{-1}$) allows unsupervised clustering algorithms to identify clusters objectively. Techniques such as DBSCAN and \textsc{HDBSCAN} can sift through the 5-dimensional phase space $(l,b,\varpi,\mu_{\alpha*},\mu_\delta)$ to detect overdensities corresponding to physical clusters (e.g. \citealt{2020A&A...635A..45C}). The advantage of these methods is that they make minimal assumptions about cluster morphology and can reveal new clusters or previously unnoticed tidal extensions of known clusters. Furthermore, by using the full covariance of Gaia astrometric uncertainties, one can assign membership probabilities rather than hard yes/no membership, yielding a more nuanced cluster population. In this thesis, we implement \textsc{HDBSCAN} on Gaia DR3 astrometric data to establish a preliminary member list for NGC 6383, identifying a tight group of co-moving stars against the dense Galactic field. This machine-learning approach is complemented by astrometric quality cuts to ensure reliability: we apply the Gaia RUWE (re-normalised unit weight error) criterion and the astrometric fidelity metric of \citet{2022MNRAS.510.2597R} to exclude stars with poor or spurious solutions (often caused by binarity or crowding) that could otherwise masquerade as cluster members.

The homogeneous \textit{Gaia} dataset also enables the correction of subtle systematic errors that previously hampered precision cluster studies. In particular, Gaia's parallaxes in DR3 carry a known zero-point offset on the order of $-17\sim\mu$as (in the sense that Gaia parallaxes are slightly too small), which varies as a function of magnitude, color, and ecliptic latitude \citep{2021A&A...649A...4L,2021A&A...649A...2L}. We account for this bias by applying the empirically derived correction from \citet{2021A&A...649A...2L} to all stars' parallaxes before distance estimation. Likewise, \citet{2021A&A...649A.124C} demonstrated a small magnitude-dependent bias in Gaia EDR3 proper motions (a $\sim 0.05$ mas yr$^{-1}$ offset between bright and faint stars due to calibration issues). We incorporate the proper-motion correction from that analysis to place bright O-stars and faint members on the same consistent reference frame. These corrections, though modest in size, are crucial for a cluster like NGC 6383: at a distance of $\sim1$ kpc, a 0.05 mas yr$^{-1}$ proper motion bias translates to a spurious velocity of $\sim0.25$ km s$^{-1}$, which could mimic or obscure physical kinematic signatures (e.g. expansion or rotation) if not removed.

With reliable membership in hand, the combination of Gaia astrometry and photometry with external surveys unlocks a comprehensive cluster characterization. Gaia DR3 provides three-band photometry ($G, G_{BP}, G_{RP}$) that, when complemented by near-infrared data (e.g. from 2MASS), spans the spectral energy distributions of cluster stars from optical to IR. This multi-band coverage is especially valuable for identifying and characterizing pre-main-sequence (PMS) stars, which often have infrared excess from circumstellar disks or exhibit distinctive positions in color-color diagrams. Indeed, specialized algorithms have been developed to exploit Gaia+2MASS data for PMS selection: one example is the Sagitta deep neural network \citep{2021AJ....162..282M}, which was trained on known star-forming regions to recognize PMS stars by their Gaia DR2 and 2MASS colours. Such tools can estimate ages of young stars and flag likely YSOs across wide fields. In this work, we use PMS classification results inspired by Sagitta (adapted to DR3) to augment our member list, cross-matching NGC 6383 candidates with published PMS catalogs and identifying dozens of YSO candidates that populate the cluster's faint end. This helps build a more complete inventory of the cluster's low-mass population, critical for assessing its initial mass function and recent star formation history.

The precision astrometry also feeds into Bayesian parameter inference for clusters. Rather than relying on traditional distance moduli from isochrone fitting alone, we can combine Gaia parallaxes with photometry in a Bayesian framework to jointly infer distance, age, and extinction. For NGC 6383, we employ a hierarchical Bayesian model (implemented in \texttt{PyMC}) that fits stellar isochrones (from the MIST models) to the observed CMD while treating the distance and extinction of the cluster as free parameters with priors informed by Gaia parallaxes \citep{2007A&A...462..157P}. This approach yields posterior probability distributions for the cluster's age, distance, and line-of-sight reddening, with realistic uncertainty estimates that account for data errors and sample selection. Importantly, it also propagates membership uncertainties: stars with lower membership probability can be probabilistically down-weighted, so they contribute less to the inference (mitigating any residual contamination). As a result, our derived cluster parameters are robust against outliers and uncertainties, providing a more solid foundation for physical interpretation. The Gaia astrometry even allows structural modeling; by converting the sky positions and distances of members to physical coordinates, we can fit models like King profiles to estimate the core radius, tidal radius, and total mass of the cluster. These structural parameters inform us about the cluster's dynamical state and can be compared to $N$-body simulations or to other clusters of similar age.

In summary, \textit{Gaia} has revolutionized open cluster studies by enabling precision membership and homogeneous analysis across the Galactic disk. The methodology adopted in this thesis - from unsupervised clustering in astrometric space, through Bayesian distance and age inference, to multi-wavelength membership validation - is a direct product of the Gaia era. We are now able to treat clusters like NGC 6383 with a level of rigor and detail that was previously impossible, ensuring that the results (membership lists, ages, etc.) are reproducible and backed by quantified uncertainties rather than subjective selection. The following sections apply this Gaia-driven approach to NGC 6383, aiming to resolve longstanding ambiguities about this cluster's parameters and to place it in context among the Milky Way's young stellar groups.

\subsection{NGC 6383: A young cluster in the Sagittarius arm}\label{sec:NGC6383}

NGC 6383 is a young open cluster located in the southern sky (constellation Scorpius) at a distance of roughly 1.1 kpc in the Carina-Sagittarius spiral arm of the Milky Way \citep{1959ApJS....4..257S}. It lies very close to the Galactic plane ($\ell \approx 355.7^\circ$, $b \approx +0.05^\circ$) in a rich star-forming complex known as Sharpless 2-12 (Sh2-012). This is a bright \ion{H}{ii} region of ionized gas, part of the Sagittarius \textsc{OB1} association that also includes the famous Lagoon Nebula cluster NGC 6530 and the cluster NGC 6531. NGC 6383 itself is embedded in an extended H II region cataloged as RCW 132, which has a crescent or shell-like morphology about $110'\times80'$ in extent. The presence of this ionized nebula signals that the cluster is very young - on the order of a few Myr - with massive stars hot enough to excite hydrogen emission. Indeed, the cluster is dominated by a central O-star system, HD 159176, which is an O7 V + O7 V double-lined spectroscopic binary. HD 159176 (visual magnitude $V\sim6.7$) is by far the most luminous member of NGC 6383 and is largely responsible for ionizing the surrounding nebula. Ultraviolet photons from this O+O binary create the Sh2-012 \ion{H}{ii} region and likely shape the surrounding molecular cloud. Because of its prominence, HD 159176 has often been regarded as the "central star" of NGC 6383, and its properties provide clues to the cluster's age and history. Spectroscopic studies indicate HD 159176 has an age of only about $2.5\pm0.5$ Myr, consistent with an early O-type main-sequence lifetime. Notably, this age is similar to the median age of low-mass pre-main-sequence stars in the cluster (around 2-4 Myr, as discussed below), suggesting a coeval formation for high- and low-mass members. On the other hand, some authors have speculated that HD 159176 could be slightly older or even a runaway that triggered the cluster's formation: \citet{1978MNRAS.182..607F} proposed that feedback from this massive binary may have initiated a secondary burst of star formation in NGC 6383's core and surrounding region, implying the O-stars might have formed just prior to the bulk of the cluster. Resolving whether HD 159176 is a true member and coeval with NGC 6383 or a precursor object is one of the intriguing questions addressed by modern data.

Previous studies of NGC 6383 over many decades have reported a wide range of values for its fundamental parameters, reflecting both the cluster's observational challenges and the evolving methodologies. Early estimates of the distance, for example, varied dramatically. \citet{1930LicOB..14..154T} gave an upper limit of about $\sim2.1$ kpc, whereas \citet{1949ApJ...110..117S} obtained a distance as low as $\sim0.75$ kpc. Such discrepancies arose because these early works lacked reliable photometric calibration and had no direct way to measure extinction or separate cluster stars from field stars in the crowded Milky Way plane. By the 1970s-1990s, photoelectric and CCD photometry enabled somewhat more consistent results: most authors converged on a distance of roughly 1.0-1.3 kpc for NGC 6383 \citep{2010A&A...511A..25R}. For instance, \citet{2007A&A...462..157P} (using Strömgren photometry) derived $d \approx 1.7$ kpc, while \citet{2018A&A...610A..30A} found $d \approx 0.84$ kpc from modeling of B-type members, and the recent catalog of \citet{2021MNRAS.504..356D} lists about $1.09\pm0.10$ kpc (consistent with the Gaia DR2 parallax of cluster stars). The current consensus, supported by Gaia DR3 parallaxes (mean $\varpi \sim 0.90$ mas), places NGC 6383 at around $1.1\pm0.1$ kpc from the Sun, which we will adopt in this work (and refine via our Bayesian analysis). The cluster's angular size on the sky is approximately $20'$ in diameter, as measured by the extent of its member star concentrations. At 1.1 kpc, this corresponds to a linear radius of about 6-7 pc, although recent Gaia-based searches suggest the cluster may have an extended halo or tidal features beyond the traditionally cataloged boundary.

The age of NGC 6383 has been a matter of debate, complicated by the dual presence of very massive stars and an abundance of low-mass YSOs. Early photometric studies classified NGC 6383 as extremely young. \citet{1978MNRAS.182..607F} estimated an age of about $2$-$3$ Myr based on the cluster's brightest stars and H$\alpha$ emission objects. Other authors found ages in the range $4$-$5$ Myr \citep[e.g.][]{1989MNRAS.236..263P,1991MNRAS.249...76B}, noting the cluster's main sequence turn-off is at spectral type B1-B2. A significant outlier was the work of \citet{1968ArA.....5....1L}, who reported a much older age of $\sim20$ Myr; however, this was likely an overestimate due to unrecognized field star contamination and has not been supported by subsequent data. Modern observations strongly indicate NGC 6383 is very young: it contains dozens of bona fide PMS stars and even deeply embedded objects. \citet{2019MNRAS.484.5102K}, using optical $ugriH\alpha$ photometry from VPHAS+ combined with Gaia DR2 astrometry, identified 55 classical T Tauri stars (CTTS) in and around NGC 6383 with a median age of $2.8\pm1.6$ Myr. They found that these CTTS are concentrated both near the cluster core and along bright nebulous rims at the periphery of Sh 2-012, suggesting a nearly coeval formation across the region \citep{2019MNRAS.484.5102K}. The presence of strong H$\alpha$ emitters with circumstellar discs confirms that star formation has been ongoing in the last few Myr. Indeed, the ages of newly formed stars in NGC 6383 span roughly 1-6 Myr, implying that while the bulk of stars likely formed in a single burst, there may be a slight age spread or sequential star formation, possibly triggered by feedback from the O-star(s). On the other hand, isochrone fits to the highest-mass members have occasionally hinted at ages up to 5-7 Myr. For example, if one of the bright B-type cluster members (often referred to as "Star No. 6" in historical studies) has evolved off the main sequence, or if HD 159176 were anomalously overluminous (as in a blue straggler scenario), the cluster could be older than the CTTS suggest. \citet{2018A&A...610A..30A} discuss this possibility, arguing that NGC 6383's age might be in the range 6-10 Myr if HD 159176 did not form contemporaneously with the lower-mass stars. However, such an age would be difficult to reconcile with the large PMS population and the general absence of a well-developed red giant population in the cluster. Our working hypothesis (to be tested with Gaia DR3 data) is that NGC 6383's age is on the order of 3-5 Myr, consistent with a young post-embedded cluster that has just emerged from its parental cloud.

The line-of-sight extinction toward NGC 6383 is moderate but significant, given its location in the Galactic plane. Earlier studies using optical photometry found color excesses around $E(B-V)\approx0.3$ mag for cluster members \citep{1971A&AS....4..241B,2008hsf2.book..497R}, though values as high as $E(B-V)\approx0.5$ mag have been reported \citep{2018A&A...610A..30A}. The consensus from multi-color photometry (e.g. \citealt{2007A&A...462..157P}) is $E(B-V)\approx0.32$ mag, which corresponds to a visual extinction $A_V\approx1.0$ mag assuming a standard reddening law. \citet{2008hsf2.book..497R} note that part of the scatter in literature values comes from differential extinction across the cluster: there is patchy dust in the Sh 2-012 region, and some stars (especially those projected near dark nebular filaments) suffer higher extinction. Rauw \& De Becker also describe a large shell of ionized and neutral gas around the cluster (the RCW 132 region) with an approximate radius of $1^\circ$, likely blown out by HD 159176's stellar wind. This shell and associated dust could account for the localized higher reddening found by \citet{2018A&A...610A..30A}. In our analysis, we use Gaia and 2MASS photometry to solve for the cluster's mean extinction and possible differential reddening, taking advantage of the fact that the Gaia DR3 data include color information that can help distinguish reddening effects from temperature in the stellar locus.

Aside from its global parameters, NGC 6383 is of particular interest for its stellar content and dynamics. X-ray observations with \textit{XMM-Newton} detected 76 point-like X-ray sources in the vicinity of the cluster (excluding HD 159176), most of which were attributed to low-mass PMS stars belonging to NGC 6383 \citep{2010A&A...511A..25R}. These X-ray-emitting young stars are concentrated toward the cluster center, reinforcing their membership, and they exhibit the elevated X-ray activity typical of T Tauri stars. Investigations of variability in NGC 6383 have found two $\beta$ Cephei-like pulsating B stars and several rotating variables, indicating that even at a few Myr, some intermediate-mass stars have evolved enough to show variability (or they could be PMS pulsators; e.g. \citealt{2005MNRAS.357..345Z}). The cluster's high-mass binary fraction is noteworthy: HD 159176 itself is a massive binary with a short orbital period (a few days), and at least one other B-type member is a spectroscopic binary. Such a prevalence of massive binaries might influence the cluster's dynamical evolution by injecting energy into the system (through three-body encounters) and potentially ejecting some members. One question we address is whether NGC 6383 shows signs of mass segregation - i.e. whether the O and B stars are more centrally concentrated than the lower-mass stars. Qualitatively, it has been noted that the brightest members (O7 and a few early B stars) lie near the cluster's center. If we find quantitatively that the cluster's massive stars have a significantly smaller average radius than the low-mass members, this could suggest primordial mass segregation (given the cluster's young age, dynamical mass segregation would not fully act yet). This ties into the cluster's relaxation state: using our estimates of the core radius and member density, we can compute the relaxation time and compare it to the cluster age. If (as expected) the relaxation timescale is longer than a few Myr, NGC 6383 should not be dynamically relaxed, meaning any mass segregation or ordered mass-dependent structure was likely inherited at birth or due to early formation processes rather than two-body relaxation. This appears to be the case - NGC 6383 has not yet achieved dynamical relaxation, consistent with it being a very young system still in the early phases of cluster evolution.

In light of the above, NGC 6383 emerges as an excellent case study of a cluster at the threshold of emerging from its nascent cloud, with ongoing star formation, a mix of massive stars and many PMS stars, and initial signs of dynamical effects but not yet fully relaxed. It bridges the gap between embedded clusters (like the Orion Nebula Cluster, which is even younger and still partially in its molecular cloud) and slightly older open clusters (like the Pleiades at 125 Myr, which is dynamically evolved and cleared of gas). By studying NGC 6383 with Gaia and complementary data, we can refine its properties and address open questions: How many members does it have and down to what mass? What is its precise age and star formation history? Is there evidence of sequential star formation (e.g. triggered by HD 159176)? What are its present-day mass function and total mass? Is it bound to remain as a cluster or already dispersing? The analysis in subsequent chapters will tackle these points, building a comprehensive picture of NGC 6383 as a benchmark young cluster in the inner Galaxy.

\subsection{Research objectives and hypotheses}

Given the context above, the primary objective of this thesis is to provide a comprehensive, data-driven characterization of NGC 6383 by combining Gaia DR3 astrometry, multi-band photometry, and advanced statistical techniques. The specific goals and questions we aim to address are:
•	Membership Census: What is the most reliable and complete membership list of NGC 6383 that can be obtained from Gaia DR3 (and 2MASS) data? By utilizing unsupervised clustering (HDBSCAN) and astrometric/photometric filters, we seek to identify cluster members across all mass ranges, including faint low-mass stars, while minimizing field contamination. A robust member catalog is the foundation for all subsequent analysis.
•	Fundamental Parameters: What are the cluster's fundamental parameters - distance, age, and overall reddening - as inferred from the combined astrometric and photometric data? We will use Bayesian isochrone fitting and Gaia parallaxes to obtain precise estimates of the heliocentric distance and age, and compare these to historical values. We hypothesize (based on preliminary evidence) that the distance is around 1.1 kpc and the age on the order of 3-5 Myr, but we will test alternative scenarios (e.g. older age hypotheses) quantitatively. We will also estimate the cluster's metallicity if possible (though Gaia photometry alone is not very sensitive to metallicity for young stars, we assume roughly solar metallicity given the cluster's location and early-type stars).
•	Cluster Structure and Dynamics: How is NGC 6383 structured spatially, and what does this imply about its dynamical state? Using the member distribution, we will derive the cluster's core radius, tidal radius, and any evidence of tidal tails or halo. We plan to fit a King model or similar to the radial density profile of members. From the spatial distribution of different mass groups, we will quantify mass segregation (e.g., via the minimum spanning tree method or comparing radial distributions). The cluster's velocity dispersion (in tangential velocities, since we have proper motions for many stars) will be used to estimate its dynamical mass and virial state. Our hypothesis is that NGC 6383, at $\sim3$ Myr, will show mild primordial mass segregation (massive stars more central) and have a sub-virial or roughly virial velocity dispersion, indicating it could be bound but not yet dramatically expanding or contracting.
•	Pre-main-sequence Population: What fraction of the cluster's members are PMS stars and YSOs, and what can we deduce about the cluster's recent star formation history? We will identify PMS candidates via their positions in the Gaia+2MASS CMD and color-color diagrams, and by cross-matching with known YSO catalogs (e.g. Spitzer or WISE infrared excess sources, H$\alpha$ emitters from VPHAS+). By placing these candidates on theoretical pre-main-sequence isochrones, we can estimate ages for individual young stars and see if there is an age spread. We anticipate finding a significant population (possibly on the order of 20-30\% of the total members) of young stars $<5$ Myr, which would confirm that NGC 6383 is an active star-forming region. If there are very young objects ($\sim1$ Myr) at the periphery (e.g. near the bright rim of the H II region), that could indicate triggered star formation by the expansion of Sh 2-012.
•	Comparison with Other Clusters: How do the properties of NGC 6383 compare with other young open clusters, and what broader implications does this have? Although not a formal objective per se, as we discuss our results we will compare NGC 6383 to clusters of similar age and mass (like NGC 6530, or the Orion Nebula Cluster) to contextualize its mass function, mass segregation, and survival prospects. We suspect NGC 6383 is somewhat less massive than massive starburst clusters, and being in the inner Galaxy, it might dissolve on shorter timescales due to stronger tidal forces. Understanding where NGC 6383 lies on the spectrum from bound young cluster to unbound association will inform theories of cluster formation and dispersal.

In formulating these objectives, our working hypotheses are: (1) NGC 6383 is a bound cluster at $d\approx1.1$ kpc with an age of $\sim3$-4 Myr, whose observed massive-star segregation is mostly primordial; (2) the cluster contains a substantial population of PMS stars (age $\sim$1-6 Myr) indicating one main epoch of star formation with possibly some triggered secondary star formation around the edges; (3) the cluster's global properties (mass $\sim10^3,M_\odot$, core radius $\sim2'$) are in line with other young clusters that will likely survive a few tens of Myr, but NGC 6383 has not yet lost its residual gas long enough to fully settle dynamically; and (4) improved data (Gaia DR3) will show that some previously assumed members (potentially including the O-star binary) might not share the exact kinematics of the cluster, prompting a reevaluation of membership for extreme objects. The subsequent analysis will test these hypotheses with quantitative evidence.

\subsection{Thesis structure}

The remainder of this thesis is organized as follows:
\begin{itemize}
    \item \textbf{Chapter 2: Background and Literature Review.} We provide an in-depth review of the theoretical and observational background relevant to this study. This includes: (i) classical and modern methods of open cluster membership determination in the Gaia era, highlighting studies that have used clustering algorithms and probabilistic models; (ii) cluster structural properties and dynamical diagnostics, covering concepts like core radius, tidal radius, mass segregation metrics, and relaxation time, with examples from literature; (iii) the properties of pre-main-sequence stars and young stellar objects, and how they are identified via emission lines, infrared excess, and Gaia kinematics (including a review of the Sagitta neural network approach and other machine-learning methods for young star identification); and (iv) Bayesian inference techniques for clusters, explaining the principles of isochrone fitting, parameter estimation with MCMC, and prior assumptions. This chapter establishes the context and justification for the methods we apply to NGC 6383.
    \item \textbf{Chapter 3: Data and Methodology.} Here we describe the data sources and the step-by-step methodology of our analysis. We detail the retrieval of Gaia DR3 data for the region of NGC 6383 and the quality cuts applied (parallax and proper-motion uncertainties, RUWE, astrometric fidelity) to define a clean initial sample. We then explain the use of the \textsc{HDBSCAN} algorithm to detect the cluster in astrometric space, including how we chose the algorithm's hyperparameters and how we combined proper motion clustering with parallax filtering to refine membership. Next, we outline the Bayesian modeling approach: constructing a likelihood function for the cluster's CMD and astrometry, setting priors for distance, age, and reddening, and running the \textsc{PyMC} sampler (with the No-U-Turn Sampler, NUTS) to obtain posterior distributions of cluster parameters. We also describe the use of \textsc{ASteCA} (Automated Stellar Cluster Analysis) and MIST isochrones for conventional fitting as a cross-check. The chapter covers how we identified PMS stars, including the application of the Sagitta model on our photometric data and the cross-match with external YSO catalogs (e.g., from infrared surveys). Lastly, we present the custom analysis pipeline (the \textsc{COSMIC} package) developed to integrate these steps, and we discuss the computational tools and libraries (Astropy, Pandas, NumPy, etc.) that ensure the analysis is reproducible.
    \item \textbf{Chapter 4: Results.} This chapter presents the results of our analysis. We start with the updated membership list of NGC 6383, giving summary statistics (number of members, spatial extent, etc.) and showing the vector point diagram (proper motion plot) and parallax distribution of the identified members to illustrate the cluster's kinematic coherence. We then provide the results of the Bayesian parameter inference: the posterior estimate of the cluster's distance (with uncertainty), age distribution, and reddening $E(B-V)$. We compare these results to literature values in a concise table. Subsequently, we report the derived structural parameters - core radius, tidal radius, and any evidence of an extended halo or fragments. If a King model fit was performed, we give the best-fit concentration parameter and discuss the cluster's likely tidal radius in physical units. The chapter then examines the Hertzsprung-Russell diagram and CMD of the cluster with the best-fit isochrone overplotted, highlighting the sequence of high-mass stars and the locus of the pre-main-sequence members. We identify which members are candidate PMS/YSOs and give an estimate of their fraction. Using multi-band photometry, we illustrate the presence of infrared excess for certain YSOs and perhaps show a spatial map of where PMS stars lie relative to the cluster center (e.g., are they centrally concentrated or in the surrounding nebula). We also analyze the kinematics of the cluster: the internal proper motion dispersion and any systematic trends (expansion or rotation). Finally, we present evidence regarding mass segregation by comparing the radial distribution of massive versus low-mass members and by computing a mass segregation index. We find, for example, that the most massive $\sim5$ stars lie within a smaller radius than expected for a random distribution, indicating mass segregation at a $>95\%$ confidence level. All results are accompanied by figures (maps, CMDs, profiles) and uncertainties, and we emphasize points of agreement or discrepancy with previous studies.
    \item \textbf{Chapter 5: Discussion.} In this chapter, we interpret our findings in the broader context of cluster formation and evolution. We first provide a revised picture of NGC 6383: summarizing its confirmed parameters and what they imply about its status (e.g., "NGC 6383 is a young (4 Myr) cluster at 1.1 kpc, moderately mass segregated and likely bound, with ongoing star formation on its periphery"). We compare our results to previous work - for instance, if our derived age is 3.5 Myr, we discuss how this sits between earlier estimates and how the inclusion of Gaia data resolved previous contradictions (such as excluding non-members that made some authors infer older ages). We also discuss interesting individual objects: for example, the status of HD 159176 (did our analysis confirm it as a cluster member kinematically, or is it an outlier?), and other notable stars like emission-line objects or binaries. We then explore the implications for star formation in NGC 6383: if we found that the low-mass stars have a similar age to HD 159176, it supports a coeval formation; if some are younger and lie near cloud edges, it supports triggered secondary star formation by the O-star's feedback. We examine whether the cluster's degree of mass segregation is likely primordial - using a simple estimate of relaxation time vs age, we can say if it's too young for dynamical segregation, hence any segregation was likely imprinted at birth. This leads to a discussion on cluster formation models (did NGC 6383 form mass-segregated or not?). We also consider the cluster's future: given its mass and the tidal radius we found, is NGC 6383 likely to remain bound for tens of Myr, or will it disperse relatively quickly? This touches on the cluster's location in the Galaxy (being near the Galactic plane in a potentially strong tidal field). Finally, we acknowledge the limitations of our study - for example, the lack of Gaia radial velocities for most members (meaning we rely on proper motions only for dynamics), potential biases due to unresolved binaries, and uncertainties in the isochrone models at very young ages. We propose how future data (e.g., Gaia DR4, deeper IR surveys, or dedicated spectroscopy) could improve the picture. This sets the stage for further inquiry and situates NGC 6383 among other well-studied clusters.
    \item \textbf{Chapter 6: Conclusions and Future Work.} The thesis concludes with a summary of the key findings and their significance. We reiterate the improved parameters for NGC 6383 (membership, distance, age, etc.) and how these contribute to the field of open cluster studies. We highlight how the combination of Bayesian methods and Gaia data provided new insights (for instance, resolving the cluster's age debate or confirming/refuting the cluster membership of the O-star binary). We discuss the broader implications - e.g., what NGC 6383 tells us about star cluster formation efficiency in the Carina-Sagittarius arm or the role of massive stars in small clusters. Finally, we outline potential future work: this could include applying our \textsc{COSMIC} pipeline to other young clusters to build a comparative sample, incorporating Gaia's upcoming radial velocity and binary solutions to deepen the dynamical analysis, or perhaps high-resolution spectroscopy of NGC 6383 members to get metallicity and a more precise age via lithium dating of PMS stars. We note that the methods developed here are broadly applicable, and extending them can lead to a better understanding of cluster evolution across the Galaxy. The chapter closes with some reflective remarks on what NGC 6383 exemplifies about young clusters and the continuing advances made possible by Gaia and complementary surveys.
\end{itemize}

The thesis is supplemented by appendices that provide supporting material, including the full membership catalog of NGC 6383 (with coordinates, magnitudes, and membership probabilities), additional figures (such as spectral energy distributions of certain YSOs or corner plots of posterior distributions from the Bayesian analysis), and technical details of the data analysis (like the exact ADQL query used to retrieve Gaia data, and a description of the computing environment and software versions to aid in reproducibility). These appendices ensure that all results in the main chapters can be traced back to data and code, aligning with the reproducible research ethos of this study.

Overall, by uniting cutting-edge data (Gaia DR3, 2MASS, etc.) with modern analysis techniques in a reproducible framework, this thesis aims to not only answer specific questions about NGC 6383 but also demonstrate an approach to studying open clusters that can be generalized to other clusters in the Gaia era. The insights gained from NGC 6383 will contribute to our general understanding of how young clusters form, evolve, and eventually disperse within our Galaxy.
\end{document}
% References (to be added to the bibliography):
% 	•	Bailer-Jones, C. A. L., Rybizki, J., Fouesneau, M., Demleitner, M., & Andrae, R. (2021). Estimating distances from parallaxes. V. Geometric and photogeometric distances to 1.47 billion stars in Gaia EDR3. AJ, 161, 147. DOI: 10.3847/1538-3881/abd806
% 	•	Cantat-Gaudin, T. & Anders, F. (2020). Painting a portrait of the Galactic disc with its stellar clusters. A&A, 640, A1. DOI: 10.1051/0004-6361/202038192
% 	•	Cantat-Gaudin, T. & Brandt, T. D. (2021). Characterizing and correcting the proper motion bias of the bright Gaia EDR3 sources. A&A, 649, A124. DOI: 10.1051/0004-6361/202140522
% 	•	Castro-Ginard, A., et al. (2020). New open clusters in the Galactic disk by harnessing the power of Gaia (DR2). A&A, 635, A45. DOI: 10.1051/0004-6361/201937386
% 	•	Collinder, P. (1931). On structural properties of open galactic clusters. Annals of the Observatory of Lund, 2, 1.
% 	•	Dias, W. S., et al. (2021). Updated parameters of 1743 open clusters based on Gaia DR2. MNRAS, 504, 356-371. DOI: 10.1093/mnras/stab770
% 	•	FitzGerald, M. P. (1978). NGC 6383 and HD 159176: a very young open cluster with an exciting binary. MNRAS, 182, 607-621. DOI: 10.1093/mnras/182.4.607
% 	•	Gaia Collaboration (Prusti, T. et al.) (2016). The Gaia mission. A&A, 595, A1. DOI: 10.1051/0004-6361/201629272
% 	•	Gaia Collaboration (Vallenari, A. et al.) (2023). Gaia Data Release 3: Summary of the content and survey properties. A&A, 674, A1. DOI: 10.1051/0004-6361/202243940
% 	•	Kalari, V. M. (2019). Classical T-Tauri stars with VPHAS+: II. NGC 6383 in Sh 2-012. MNRAS, 484, 5102-5112. DOI: 10.1093/mnras/stz250
% 	•	Kharchenko, N. V., Piskunov, A. E., Röser, S., Schilbach, E., & Scholz, R.-D. (2005). Astrophysical parameters of Galactic open clusters. A&A, 438, 1163-1173. DOI: 10.1051/0004-6361:20042523
% 	•	Lindoff, U. (1968). Age determination of open clusters and their relation to the spiral structure of the Galaxy. Arkiv for Astronomi, 5, 1-89.
% 	•	Lindegren, L., et al. (2021). Gaia EDR3: The astrometric solution and parallax bias. A&A, 649, A2. DOI: 10.1051/0004-6361/202039709
% 	•	McBride, A., et al. (2021). Photometric search for pre-main-sequence stars with deep learning (Sagitta). AJ, 162, 282. DOI: 10.3847/1538-3881/ac2432
% 	•	Morales, E. F. E., et al. (2013). Catalog of stellar clusters in the inner Galaxy (l= 300° to 60°). A&A, 560, A76. DOI: 10.1051/0004-6361/201322177
% 	•	Paunzen, E., Netopil, M., & Zwintz, K. (2007). Investigating star formation in the young open cluster NGC 6383. A&A, 462, 157-162. DOI: 10.1051/0004-6361:20065513
% 	•	Perry, C. L., Hill, G., & Christodoulou, D. M. (1989). Photoelectric photometry of open clusters. III. NGC 6383. MNRAS, 236, 263-270. DOI: 10.1093/mnras/236.2.263
% 	•	Pulgar-Escobar, L. M., et al. (2025). Characterizing NGC 6383: A study of pre-main-sequence stars, mass segregation, and age using Gaia DR3 and 2MASS. A&A, submitted (arXiv:2405.09145).
% 	•	Rauw, G., & De Becker, M. (2008). The multiwavelength picture of star formation in the very young open cluster NGC 6383. In B. Reipurth (Ed.), Handbook of Star Forming Regions, Vol. II: The Southern Sky (ASP Monograph, Vol. 5, p. 497). (arXiv:0808.3887)
% 	•	Rauw, G., Manfroid, J., & De Becker, M. (2010). A photometric and spectroscopic investigation of star formation in NGC 6383. A&A, 511, A25. DOI: 10.1051/0004-6361/200912780
% 	•	Rybizki, J., et al. (2022). A classifier for spurious astrometric solutions in Gaia EDR3 (astrometric fidelity). MNRAS, 510, 2597-2612. DOI: 10.1093/mnras/stab3586
% 	•	Sharpless, S. (1959). A catalogue of H II regions. ApJS, 4, 257. DOI: 10.1086/190049
% 	•	Shapley, H. (1949). Note on some distant open clusters. ApJ, 110, 117-122. DOI: 10.1086/145200
% 	•	Thé, P. S., Tjin A Djie, H. R. E., & van Esch, B. P. M. (1985). A photometric study of the young open cluster NGC 6383. A&A, 151, 391-394.
% 	•	Trumpler, R. J. (1930). Preliminary results on the distances, dimensions, and space distribution of open star clusters. Lick Obs. Bull., 14, 154-188.
% 	•	Zwintz, K., et al. (2005). Search for pulsating pre-main-sequence stars in NGC 6383. MNRAS, 357, 345-353. DOI: 10.1111/j.1365-2966.2005.08655.x